\title{Shannon-Fano coding}
\author{Vicente González Ruiz}
\maketitle
\tableofcontents

\section{History}
\begin{itemize}
\tightlist
\item
  At the end of the 40's, Claude E. Shannon (Bell Labs) and R.M. Fano
  (MIT) developed the Shannnon-Fano codec~\cite{fano1949transmission,
    shannon1948mathematical}.
\end{itemize}

\section{Encoder}
\begin{enumerate}
\tightlist
\item
  Sort the symbols using their probabilities.
\item
  Split the set of symbols into two subsets in a way in which the each
  subset have the same total probability.
\item
  Assign a different bit to each set.
\item
  Repeat the previous procedure to each subset until each subset has
  only one symbol.
\end{enumerate}

\subsection{Example}
\begin{itemize}
\tightlist
\item
  Let's use the following probabilistic model:
\end{itemize}

\svgfig{graphics/shannon-fano_example}{2cm}{200}

Using it, this is the Shannon-Fano coding:

\svgfig{graphics/shannon-fano_example-coding}{10cm}{1000}

\subsection{Decoder}
TO-DO.

\bibliography{text-compression}
